\chapter{\ifproject%
\ifenglish Experimentation and Results\else การทดลองและผลลัพธ์\fi
\else%
\ifenglish System Evaluation\else การประเมินระบบ\fi
\fi}

ในบทนี้จะทดสอบและประเมินผลการทำงานของระบบ โดยเน้นที่ฟังก์ชันหลักที่ได้พัฒนา 
เพื่อพิสูจน์ว่าระบบสามารถตอบสนองความต้องการของผู้ใช้และผู้ขายได้อย่างเหมาะสม \cite{pressman2014}

\section{การทดสอบฟังก์ชันหลัก (Core Functionality Testing)}
ได้ทำการทดสอบฟังก์ชันต่าง ๆ ของระบบ ดังนี้ \cite{ecommerce2019}
\begin{itemize}
  \item \textbf{การลงทะเบียนและเข้าสู่ระบบ}  
  ผู้ใช้สามารถสมัครสมาชิกและเข้าสู่ระบบได้อย่างถูกต้อง รวมถึงการยืนยันตัวตน
  \item \textbf{การจัดการสินค้า}  
  ผู้ขายสามารถเพิ่ม ลบ และแก้ไขข้อมูลสินค้าได้
  \item \textbf{แดชบอร์ดผู้ขาย}  
  แสดงข้อมูลยอดขาย สินค้าขายดี และการแจ้งเตือนสินค้าคงเหลือได้ตามจริง
  \item \textbf{การแนะนำสินค้าเชิงพฤติกรรม}  
  ระบบสามารถสุ่มแบบสอบถามและนำเสนอสินค้าที่เหมาะสมกับผู้ใช้
  \item \textbf{การชำระเงิน}  
  ผู้ใช้สามารถเลือกชำระเงินผ่าน PromptPay ได้สำเร็จและมีการบันทึกข้อมูลการชำระเงิน
\end{itemize}

\section{การทดสอบการใช้งาน (Usability Testing)}
ได้ทำการทดสอบกับกลุ่มผู้สูงอายุ 10 คน โดยให้ใช้งานระบบจริง \cite{otop2020}
ผลการประเมินพบว่า
\begin{itemize}
  \item ผู้ใช้ส่วนใหญ่สามารถใช้งานได้โดยไม่ต้องอธิบายเพิ่มเติม
  \item ขนาดตัวอักษรและการจัดวางปุ่มช่วยให้การใช้งานง่ายขึ้น
  \item มีข้อเสนอแนะให้เพิ่มปุ่มลัดสำหรับการเข้าถึงสินค้าโปรด
\end{itemize}

\section{ผลการทดลอง (Results)}
จากการทดสอบ พบว่าระบบสามารถทำงานได้ตรงตามวัตถุประสงค์ที่วางไว้ โดยสรุปดังนี้ \cite{pressman2014}
\begin{enumerate}
  \item ระบบสามารถทำงานได้เสถียรและรองรับการใช้งานหลายผู้ใช้พร้อมกัน
  \item การแสดงข้อมูลในแดชบอร์ดช่วยให้ผู้ขายตัดสินใจได้ง่ายขึ้น
  \item ระบบแนะนำสินค้าเพิ่มความน่าสนใจและสร้างประสบการณ์ที่ดีให้ผู้บริโภค
\end{enumerate}

\section{การประเมินผล (Evaluation)}
เพื่อยืนยันประสิทธิภาพ ได้ใช้ตัวชี้วัดดังนี้ \cite{ecommerce2019}
\begin{itemize}
  \item \textbf{Response Time}: การตอบสนองของระบบเฉลี่ยอยู่ที่ต่ำกว่า 2 วินาที
  \item \textbf{Accuracy}: การแสดงข้อมูลการขายและการแจ้งเตือนสินค้าคงเหลือถูกต้อง 100\%
  \item \textbf{User Satisfaction}: ผู้ใช้ให้คะแนนความพึงพอใจโดยเฉลี่ย 4.5/5
\end{itemize}

\section{สรุปผลการทดลอง}
การทดสอบยืนยันได้ว่าระบบที่พัฒนาสามารถใช้งานได้จริง ตรงตามวัตถุประสงค์ \cite{pressman2014}
และช่วยสนับสนุนผู้ประกอบการ OTOP ให้เข้าถึงตลาดออนไลน์ได้มีประสิทธิภาพมากขึ้น \cite{otop2020}
