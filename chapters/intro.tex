\chapter{\ifenglish Introduction\else บทนำ\fi}

\section{\ifenglish Project rationale\else ที่มาของโครงงาน\fi}
โครงการนี้เกิดจากการเล็งเห็นว่า ผู้ประกอบการ OTOP ในระดับชุมชนยังขาดแพลตฟอร์มกลาง 
ในการนำเสนอและจำหน่ายสินค้าอย่างทั่วถึง โดยเฉพาะกลุ่มผู้สูงอายุที่ไม่ถนัดด้านเทคโนโลยี  
ดังนั้นจึงได้พัฒนาเว็บไซต์ที่ใช้งานง่าย และมีระบบแดชบอร์ดช่วยผู้ขายในการวิเคราะห์ข้อมูลการขาย 
เช่น สินค้าขายดี จำนวนผู้เข้าชม และการแจ้งเตือนสินค้าคงเหลือ  
เพื่อเพิ่มโอกาสในการเข้าถึงตลาดและพัฒนาศักยภาพการแข่งขัน

ระบบ OTOP ในประเทศไทยมีการพัฒนาอย่างต่อเนื่อง \cite{otop2020}

\section{\ifenglish Objectives\else วัตถุประสงค์ของโครงงาน\fi}
\begin{enumerate}
    \item พัฒนาเว็บไซต์สำหรับผู้ประกอบการ OTOP ในการนำเสนอและจำหน่ายสินค้า
    \item ออกแบบแดชบอร์ดสำหรับผู้ขายเพื่อช่วยในการวิเคราะห์ข้อมูลการขาย
    \item สร้างระบบแนะนำสินค้าเชิงพฤติกรรมเพื่อเพิ่มประสบการณ์การซื้อของผู้บริโภค
    \item อำนวยความสะดวกแก่ผู้ใช้งานสูงวัยด้วยการออกแบบที่เรียบง่ายและชัดเจน
\end{enumerate}

\section{\ifenglish Project scope\else ขอบเขตของโครงงาน\fi}

\subsection{\ifenglish Hardware scope\else ขอบเขตด้านฮาร์ดแวร์\fi}
ผู้ใช้งานสามารถเข้าถึงระบบผ่านอุปกรณ์ที่รองรับอินเทอร์เน็ต เช่น คอมพิวเตอร์ สมาร์ตโฟน 
และแท็บเล็ต โดยไม่จำเป็นต้องติดตั้งอุปกรณ์เพิ่มเติม

\subsection{\ifenglish Software scope\else ขอบเขตด้านซอฟต์แวร์\fi}
โครงการครอบคลุมการพัฒนาเว็บไซต์สำหรับผู้ซื้อและผู้ขาย ระบบแดชบอร์ดสำหรับผู้ขาย 
และระบบแนะนำสินค้า โดยมุ่งเน้นการใช้งานง่าย รองรับผู้ใช้งานสูงวัย และการแสดงผลบนหลายอุปกรณ์

\section{\ifenglish Expected outcomes\else ประโยชน์ที่ได้รับ\fi}
\begin{itemize}
    \item ผู้ประกอบการ OTOP สามารถขายสินค้าได้สะดวกขึ้นและเข้าถึงผู้บริโภคมากขึ้น
    \item ผู้ซื้อ โดยเฉพาะผู้สูงอายุ สามารถเลือกซื้อสินค้าได้ง่ายผ่านแพลตฟอร์มที่เป็นมิตร
    \item ระบบแดชบอร์ดช่วยผู้ขายวางแผนการตลาดและบริหารจัดการสต็อกได้มีประสิทธิภาพ
    \item สร้างช่องทางสนับสนุนเศรษฐกิจชุมชนและส่งเสริมสินค้า OTOP
\end{itemize}

\section{\ifenglish Technology and tools\else เทคโนโลยีและเครื่องมือที่ใช้\fi}

\subsection{\ifenglish Hardware technology\else เทคโนโลยีด้านฮาร์ดแวร์\fi}
เซิร์ฟเวอร์สำหรับให้บริการเว็บไซต์ และอุปกรณ์ของผู้ใช้ เช่น คอมพิวเตอร์และสมาร์ตโฟน \cite{latexcompanion}

\subsection{\ifenglish Software technology\else เทคโนโลยีด้านซอฟต์แวร์\fi}
ซอฟต์แวร์และเครื่องมือที่ใช้ในการพัฒนาโครงการนี้ประกอบด้วย \cite{responsiveweb}

\begin{itemize}
    \item \textbf{Frontend:} React.js ร่วมกับ Tailwind CSS 
    เพื่อสร้างส่วนติดต่อผู้ใช้ที่เรียบง่าย ใช้งานสะดวก และเป็นมิตรกับผู้สูงอายุ \cite{responsiveweb}
    \item \textbf{Backend:} Next.js (Node.js + TypeScript) 
    รองรับทั้ง Server-side Rendering (SSR) และ API สำหรับการติดต่อกับฐานข้อมูล
    \item \textbf{Database:} PostgreSQL 
    ใช้จัดเก็บข้อมูลผู้ใช้ สินค้า คำสั่งซื้อ และข้อมูลเชิงวิเคราะห์
    \item \textbf{Authentication:} OAuth 2.0 และ JWT 
    เพื่อจัดการสิทธิ์การเข้าถึงและความปลอดภัยของผู้ใช้งาน
    \item \textbf{Payment:} ระบบการชำระเงินผ่าน PromptPay 
    สะดวกและเหมาะสมกับผู้ใช้ในประเทศไทย
    \item \textbf{Development Tools:} 
    GitHub สำหรับจัดการซอร์สโค้ดและการทำงานร่วมกัน, 
    Docker สำหรับการจัดการสภาพแวดล้อมการพัฒนา \cite{docker2023}, 
    และ Visual Studio Code สำหรับการพัฒนาโปรแกรม
    \item \textbf{Accessibility:} การออกแบบตามมาตรฐาน WCAG 2.1 
    เพื่อรองรับผู้ใช้งานทุกกลุ่ม รวมถึงผู้สูงอายุและผู้พิการทางสายตา
\end{itemize}

\section{\ifenglish Project plan\else แผนการดำเนินงาน\fi}
\begin{plan}{6}{2025}{2}{2026}
    \planitem{6}{2025}{7}{2025}{ศึกษาความต้องการและเก็บข้อมูลจากผู้ใช้งาน OTOP และผู้สูงอายุ}
    \planitem{7}{2025}{8}{2025}{ออกแบบโครงสร้างระบบและฐานข้อมูลเบื้องต้น}
    \planitem{8}{2025}{10}{2025}{ออกแบบ UI/UX ของเว็บไซต์และแดชบอร์ด}
\end{plan}

\section{\ifenglish Roles and responsibilities\else บทบาทและความรับผิดชอบ\fi}
ในการทำงานได้มีการแบ่งบทบาทของนักศึกษาออกเป็นกลุ่ม ได้แก่  
\begin{itemize}
    \item ฝ่ายวิเคราะห์และเก็บข้อมูลความต้องการผู้ใช้
    \item ฝ่ายออกแบบระบบและ UI/UX
    \item ฝ่ายพัฒนาและทดสอบระบบ (Frontend/Backend)
    \item ฝ่ายจัดทำเอกสารและนำเสนอผล
\end{itemize}
ซึ่งแต่ละฝ่ายต้องใช้ความรู้ด้านวิศวกรรมคอมพิวเตอร์ 
เช่น การเขียนโปรแกรม การออกแบบฐานข้อมูล และการจัดการโครงการซอฟต์แวร์

\section{\ifenglish%
Impacts of this project on society, health, safety, legal, and cultural issues
\else%
ผลกระทบด้านสังคม สุขภาพ ความปลอดภัย กฎหมาย และวัฒนธรรม
\fi}
โครงการนี้ช่วยส่งเสริมการตลาดของผู้ประกอบการ OTOP และสร้างรายได้ให้แก่ชุมชน  
ผู้บริโภค โดยเฉพาะผู้สูงอายุ ได้รับประสบการณ์การใช้งานที่สะดวกและปลอดภัย  
ในด้านสังคม ช่วยกระตุ้นเศรษฐกิจฐานรากและส่งเสริมวัฒนธรรมท้องถิ่น  
ในด้านกฎหมายและความปลอดภัย ระบบได้ออกแบบโดยคำนึงถึงมาตรฐานการปกป้องข้อมูลส่วนบุคคล  
ซึ่งสอดคล้องกับแนวทางการใช้เทคโนโลยีอย่างรับผิดชอบต่อสังคม
