\chapter{\ifproject%
\ifenglish Project Structure and Methodology\else โครงสร้างและขั้นตอนการทำงาน\fi
\else%
\ifenglish Project Structure\else โครงสร้างของโครงงาน\fi
\fi
}

ในบทนี้จะกล่าวถึงหลักการ ขั้นตอนการทำงาน และโครงสร้างของระบบ
ซึ่งครอบคลุมทั้งการออกแบบ การพัฒนา และการทดสอบระบบ \cite{pressman2014}
เพื่อให้ผู้อ่านเข้าใจวิธีการทำงานของโครงการอย่างเป็นระบบ

\section{ภาพรวมของระบบ (System Overview)}
ระบบเว็บไซต์และแดชบอร์ดผู้ขาย OTOP ประกอบด้วยสองส่วนหลัก ได้แก่ \cite{ecommerce2019}
\begin{itemize}
  \item \textbf{Frontend} ส่วนติดต่อผู้ใช้ที่ออกแบบให้เรียบง่าย 
  รองรับหลักการ WCAG 2.1 เช่น การเลือกสีที่มีคอนทราสต์สูง 
  การปรับขนาดตัวอักษรได้ และการนำทางด้วยคีย์บอร์ด 
  เหมาะสมกับผู้สูงอายุและผู้ใช้ทุกกลุ่ม
  \item \textbf{Backend และแดชบอร์ด} สำหรับผู้ขาย ใช้จัดการสินค้า วิเคราะห์ข้อมูลยอดขาย 
  และแจ้งเตือนสินค้าคงเหลือ
\end{itemize}

\section{กระบวนการพัฒนาโครงการ (Project Development Process)}
โครงการนี้ใช้หลักการของ \textit{Software Development Life Cycle (SDLC)} \cite{pressman2014}
โดยเลือกใช้รูปแบบการพัฒนาแบบ \textit{Agile} เพื่อความยืดหยุ่น \cite{pressman2014}
และสามารถปรับปรุงแก้ไขตามความต้องการของผู้ใช้ได้อย่างต่อเนื่อง

\subsection{ขั้นตอนการดำเนินงาน}
\begin{enumerate}
  \item \textbf{การวิเคราะห์ความต้องการ (Requirement Analysis)} 
  เก็บข้อมูลจากผู้ประกอบการ OTOP และผู้บริโภค \cite{otop2020}
  โดยคำนึงถึงข้อจำกัดของผู้สูงอายุและผู้พิการทางสายตา
  \item \textbf{การออกแบบระบบ (System Design)} 
  แยกโครงสร้าง Frontend/Backend/ฐานข้อมูล พร้อมกำหนดอินเตอร์เฟซ 
  โดยยึดหลัก WCAG 2.1 เช่น การใช้สีที่อ่านง่าย ข้อความมีคำอธิบายชัดเจน 
  และใส่คำอธิบายรูปภาพ (alt text) \cite{pressman2014}
  \item \textbf{การพัฒนาและทดสอบ (Implementation and Testing)} 
  พัฒนาโมดูลและทดสอบเชิงหน่วย/เชิงรวม \cite{pressman2014}
  โดยเสริมการทดสอบ Accessibility
  \item \textbf{การปรับปรุงและส่งมอบ (Deployment)} 
  นำระบบใช้งานจริงและปรับตามข้อเสนอแนะ \cite{ecommerce2019}
\end{enumerate}

\subsection{แผนภาพโครงสร้างระบบ (System Architecture Diagram)}
\begin{figure}[h]
  \centering
  % ----- Placeholder box แทนรูป เพื่อไม่ให้ error ถ้าไม่มีไฟล์รูป -----
  \fbox{\parbox{0.8\textwidth}{\centering
  \vspace{18mm} \textit{System Architecture Diagram Placeholder} \vspace{18mm}}}
  \caption[System architecture]{ภาพรวมโครงสร้างระบบเว็บไซต์และแดชบอร์ดผู้ขาย OTOP}
  \label{fig:system-architecture}
\end{figure}

\section{โครงสร้างฐานข้อมูล (Database Structure)}
ฐานข้อมูลเก็บข้อมูลผู้ใช้ สินค้า คำสั่งซื้อ การชำระเงิน และบันทึกสถิติการเข้าชม \cite{pressman2014}
โดยคำนึงถึงการทำ Normalization เพื่อลดความซ้ำซ้อนและเพิ่มประสิทธิภาพการสืบค้น

\subsection{ความสัมพันธ์ของข้อมูล (Entity Relationship)}
\begin{itemize}
  \item \textbf{Users}: บัญชีผู้ซื้อ/ผู้ขาย และสิทธิ์การใช้งาน \cite{pressman2014}
  \item \textbf{Products}: รายการสินค้า คุณลักษณะ ราคา และสถานะสต็อก 
  รวมถึงข้อมูลคำอธิบายรูปภาพ (alt text) เพื่อรองรับการเข้าถึงตาม WCAG \cite{ecommerce2019}
  \item \textbf{Orders}: ใบสั่งซื้อ รายการสินค้าในคำสั่งซื้อ และสถานะการจัดส่ง \cite{ecommerce2019}
  \item \textbf{Payments}: หลักฐาน/สถานะการชำระเงิน (เช่น PromptPay) \cite{promptpay2021}
  \item \textbf{Analytics}: บันทึกการเข้าชม การคลิก และสถิติยอดขาย \cite{otop2020}
\end{itemize}

\section{วิธีการทดสอบระบบ (System Testing Methodology)}

การทดสอบระบบเป็นขั้นตอนสำคัญในการตรวจสอบความถูกต้องและความสมบูรณ์ของระบบ 
เพื่อให้มั่นใจว่าทุกฟังก์ชันทำงานได้ตามที่ออกแบบไว้และเหมาะสมกับกลุ่มผู้ใช้งานเป้าหมาย 
โดยโครงการนี้ได้ใช้วิธีการทดสอบหลักดังต่อไปนี้

\begin{itemize}
  \item \textbf{Unit Testing:}  
  ทดสอบฟังก์ชันย่อยภายในระบบ เช่น  
  การเพิ่มสินค้าใหม่, การแก้ไขข้อมูลสินค้า, การบันทึกคำสั่งซื้อ, 
  และการบันทึกการชำระเงินผ่าน PromptPay ให้แน่ใจว่าทำงานได้ถูกต้อง

  \item \textbf{Integration Testing:}  
  ทดสอบการทำงานร่วมกันของแต่ละโมดูล เช่น  
  กระบวนการ “สั่งซื้อสินค้า → ชำระเงินผ่าน PromptPay → อัปเดตสถานะคำสั่งซื้อ → ลดสต็อกสินค้า”  
  เพื่อให้แน่ใจว่าข้อมูลเชื่อมโยงกันอย่างถูกต้องระหว่าง Frontend, Backend และฐานข้อมูล

  \item \textbf{User Acceptance Testing (UAT):}  
  เชิญผู้ใช้จริง เช่น ผู้ประกอบการ OTOP และผู้สูงอายุ มาทดลองใช้งานระบบจริง  
  โดยเฉพาะการใช้งานผ่านโทรศัพท์มือถือ ซึ่งเป็นอุปกรณ์หลักของกลุ่มเป้าหมาย  
  เพื่อประเมินความง่ายในการใช้งาน ความเข้าใจของเมนู และความชัดเจนของข้อมูลในแดชบอร์ด

  \item \textbf{Accessibility Testing:}  
  ทดสอบตามหลัก WCAG 2.1 โดยเฉพาะการแสดงผลบนหน้าจอขนาดต่าง ๆ  
  และกับผู้สูงอายุที่อาจมีปัญหาด้านสายตา เช่น  
  \begin{itemize}
      \item การตรวจสอบคอนทราสต์ของสีพื้นหลังและข้อความ  
      \item การปรับขนาดตัวอักษรและปุ่มกดให้เหมาะสมกับผู้สูงอายุ  
      \item การตรวจสอบว่าเว็บไซต์สามารถใช้งานด้วยคีย์บอร์ดได้ครบทุกส่วน  
      \item การทดสอบการอ่านออกเสียงด้วยโปรแกรม Screen Reader  
  \end{itemize}
\end{itemize}

การทดสอบทั้งหมดนี้ช่วยให้มั่นใจว่าระบบสามารถให้ประสบการณ์ใช้งานที่ดี 
มีความปลอดภัย และเป็นมิตรกับผู้ใช้งานทุกกลุ่ม โดยเฉพาะผู้สูงอายุซึ่งเป็นกลุ่มเป้าหมายหลักของโครงการ


\section{สรุป}
จากโครงสร้างและขั้นตอนที่กำหนด โครงการมุ่งพัฒนาแพลตฟอร์มที่ใช้งานง่าย \cite{ecommerce2019}
เหมาะกับผู้สูงอายุ และช่วยผู้ประกอบการ OTOP บริหารการขายได้มีประสิทธิภาพ \cite{otop2020}
โดยยึดหลัก WCAG 2.1 เพื่อให้ระบบเข้าถึงได้สำหรับผู้ใช้งานทุกกลุ่ม 
รวมถึงผู้พิการทางสายตาและผู้สูงอายุ
