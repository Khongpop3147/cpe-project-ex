\chapter{\ifenglish Conclusions and Discussions\else บทสรุปและข้อเสนอแนะ\fi}

\section{\ifenglish Conclusions\else สรุปผล\fi}
จากการพัฒนาระบบเว็บไซต์และแดชบอร์ดสำหรับผู้ขาย OTOP 
พบว่าระบบสามารถตอบสนองความต้องการของผู้ใช้งานได้เป็นอย่างดี 
โดยเฉพาะการช่วยผู้ขายในการจัดการสินค้า วิเคราะห์ข้อมูลยอดขาย 
และการแสดงข้อมูลเชิงพฤติกรรมเพื่อแนะนำสินค้าให้แก่ผู้บริโภค  
ระบบยังถูกออกแบบให้ใช้งานง่าย เหมาะสำหรับกลุ่มผู้สูงอายุที่ไม่ถนัดด้านเทคโนโลยี

อย่างไรก็ตาม โครงการยังมีข้อจำกัดบางประการ เช่น
\begin{itemize}
  \item ยังไม่สามารถเชื่อมต่อกับระบบคลังสินค้าภายนอกหรือระบบขนส่งได้
  \item ระบบแนะนำสินค้ายังอยู่ในระดับพื้นฐาน ใช้แบบสอบถามสั้น ๆ เท่านั้น
  \item ยังไม่สามารถยืนยันได้ว่าผู้ประกอบการ OTOP ทุกกลุ่มจะเข้าร่วมใช้งาน
\end{itemize}

\section{\ifenglish Challenges\else ปัญหาที่พบและแนวทางการแก้ไข\fi}
ในการดำเนินโครงการพบปัญหาหลัก ๆ ดังนี้
\begin{enumerate}
  \item \textbf{ข้อจำกัดด้านข้อมูล}: การเก็บข้อมูลจากผู้ขายบางรายไม่ครบถ้วน  
  \textit{แนวทางแก้ไข}: ออกแบบระบบให้มีแบบฟอร์มบังคับกรอกข้อมูลที่จำเป็น
  \item \textbf{ความยากในการออกแบบสำหรับผู้สูงอายุ}: ขนาดตัวอักษรและปุ่มบางส่วนยังเล็กเกินไป  
  \textit{แนวทางแก้ไข}: ปรับปรุง UI ให้เรียบง่ายขึ้น และรองรับการปรับขนาดอักษร
  \item \textbf{การทดสอบระบบจริง}: การทดสอบยังจำกัดในกลุ่มตัวอย่างขนาดเล็ก  
  \textit{แนวทางแก้ไข}: วางแผนทำการทดสอบเชิงปริมาณกับกลุ่มผู้ใช้จำนวนมากขึ้น
\end{enumerate}

\section{\ifenglish Suggestions and further improvements\else ข้อเสนอแนะและแนวทางการพัฒนาต่อ\fi}
ข้อเสนอแนะเพื่อพัฒนาระบบในอนาคต ได้แก่
\begin{itemize}
  \item เพิ่มระบบจัดการสต็อกสินค้าเชื่อมกับระบบคลังสินค้าและการขนส่ง
  \item พัฒนา Recommendation System ให้รองรับ AI/ML เพื่อให้คำแนะนำที่แม่นยำขึ้น
  \item ขยายระบบสู่แพลตฟอร์มสมาร์ตโฟน เพื่อให้ผู้สูงอายุใช้งานสะดวกยิ่งขึ้น
  \item เพิ่มระบบรีวิวและคะแนนสินค้า เพื่อสร้างความน่าเชื่อถือให้แก่ผู้ขาย
  \item ศึกษาและทดสอบระบบกับผู้ใช้จำนวนมากขึ้นเพื่อเก็บข้อมูลเชิงสถิติที่ชัดเจน
\end{itemize}
