\chapter{\ifenglish Background Knowledge and Theory\else ทฤษฎีที่เกี่ยวข้อง\fi}

การทำโครงงาน เริ่มต้นด้วยการศึกษาค้นคว้าทฤษฎีที่เกี่ยวข้อง 
รวมถึงงานวิจัยและโครงงานที่เคยมีผู้นำเสนอไว้แล้ว 
เพื่อสร้างความเข้าใจและเป็นพื้นฐานในการพัฒนาระบบ 
เนื้อหาในบทนี้จึงอธิบายทฤษฎี แนวคิด และความรู้ที่เกี่ยวข้อง \cite{pressman2014}
ซึ่งช่วยให้ผู้อ่านเข้าใจโครงงานในบทถัด ๆ ไปได้ง่ายขึ้น

\section{ระบบที่มีใช้อยู่ในปัจจุบัน (Existing Systems)}
ปัจจุบันมีแพลตฟอร์ม E-Commerce และเว็บไซต์ที่เกี่ยวข้องกับการจำหน่ายสินค้า OTOP อยู่แล้ว 
เช่น Shopee, Lazada และ OTOP Today โดยแต่ละระบบมีจุดเด่นและข้อจำกัดดังนี้

\begin{itemize}
    \item \textbf{Shopee/Lazada:} 
    มีระบบซื้อขายครบวงจร ฐานผู้ใช้ขนาดใหญ่ และระบบแนะนำสินค้าอัตโนมัติ 
    แต่มีความซับซ้อนเกินไปสำหรับผู้สูงอายุ และไม่เจาะจงกับผู้ประกอบการ OTOP
    \item \textbf{OTOP Today:} 
    เป็นเว็บไซต์ที่รวมสินค้า OTOP จากกรมการพัฒนาชุมชน 
    แต่ยังขาดระบบแดชบอร์ดผู้ขายและการวิเคราะห์ข้อมูลเชิงลึก
\end{itemize}

ดังนั้น โครงงานนี้จึงมุ่งพัฒนา \textbf{แพลตฟอร์มที่ออกแบบเฉพาะสำหรับผู้ประกอบการ OTOP} 
โดยเน้นอินเทอร์เฟซที่เรียบง่ายตามหลัก WCAG 2.1 และมีแดชบอร์ดสำหรับการวิเคราะห์ข้อมูล 
ซึ่งแตกต่างจากระบบที่มีอยู่ในปัจจุบัน

\section{การพัฒนาเว็บไซต์เชิงพาณิชย์ (E-Commerce Website Development)}
โครงการนี้เกี่ยวข้องกับการสร้างเว็บไซต์ที่ทำหน้าที่เป็นตลาดกลาง 
ในการจำหน่ายสินค้า OTOP จึงต้องศึกษาทฤษฎีและแนวทางการออกแบบเว็บไซต์เชิงพาณิชย์ \cite{ecommerce2019} 
ซึ่งรวมถึงหลักการออกแบบส่วนติดต่อผู้ใช้ (UI) และประสบการณ์ผู้ใช้ (UX) \cite{pressman2014} 
เพื่อให้ผู้ใช้งาน โดยเฉพาะกลุ่มผู้สูงอายุ สามารถใช้งานได้ง่าย 
มีความชัดเจน และเข้าถึงข้อมูลได้สะดวก

\textbf{ข้อแตกต่างจากระบบปัจจุบัน:}  
เว็บไซต์ทั่วไปเช่น Shopee/Lazada เน้นฟีเจอร์หลากหลายแต่ซับซ้อน 
โครงงานนี้จะออกแบบ UI ที่เรียบง่าย ใช้ฟอนต์และปุ่มขนาดใหญ่เพื่อรองรับผู้สูงอายุ 
และเจาะจงกับสินค้า OTOP

\section{ระบบแดชบอร์ดและการวิเคราะห์ข้อมูล (Dashboard and Data Analytics)}
แดชบอร์ด (Dashboard) เป็นเครื่องมือสำคัญที่ช่วยสรุปข้อมูลและแสดงผลในรูปแบบที่เข้าใจง่าย \cite{ecommerce2019} 
เช่น กราฟ แผนภูมิ หรือการแจ้งเตือน โดยโครงการนี้นำแนวคิดการวิเคราะห์ข้อมูล (Data Analytics) \cite{otop2020} 
มาใช้เพื่อแสดงข้อมูลยอดขาย สินค้าขายดี จำนวนผู้เข้าชม และการแจ้งเตือนสินค้าคงเหลือ 
เพื่อช่วยผู้ขายวางแผนการตลาดและจัดการสต็อกได้มีประสิทธิภาพมากขึ้น

\textbf{ข้อแตกต่างจากระบบปัจจุบัน:}  
OTOP Today ไม่มีแดชบอร์ดสำหรับผู้ขาย โครงงานนี้จะเพิ่มฟีเจอร์ดังกล่าว 
เพื่อให้ผู้ประกอบการเห็นข้อมูลเชิงลึกและใช้วางกลยุทธ์ได้

\subsection{การวิเคราะห์เชิงพฤติกรรม (Behavioral Analytics)}
ระบบแนะนำสินค้า (Recommendation System) ในโครงการนี้อิงจากการวิเคราะห์เชิงพฤติกรรม \cite{ecommerce2019} 
ซึ่งทำงานผ่านแบบสอบถามสั้น ๆ ที่ประเมินอารมณ์หรือความต้องการของผู้ใช้งาน 
จากนั้นระบบจะนำเสนอสินค้าที่เหมาะสม เช่น เครื่องประดับสีแดงเพื่อเสริมพลังบวก 
ถือเป็นการนำแนวคิด \textit{Personalization} มาใช้เพื่อเพิ่มคุณค่าแก่ผู้บริโภค

\subsubsection{ระบบแนะนำสินค้า (Recommendation System)}
มีหลายเทคนิคที่ใช้พัฒนา เช่น
\begin{itemize}
    \item \textbf{Content-based Filtering} ใช้ข้อมูลคุณลักษณะของสินค้าและความสนใจผู้ใช้
    \item \textbf{Collaborative Filtering} ใช้พฤติกรรมของผู้ใช้งานที่มีความคล้ายคลึงกัน
    \item \textbf{Hybrid Approach} ผสมผสานทั้งสองวิธีเพื่อเพิ่มความแม่นยำ
\end{itemize}
โครงการนี้เลือกใช้แบบสอบถามเชิงพฤติกรรมเป็นพื้นฐาน \cite{ecommerce2019} 
เนื่องจากเหมาะกับกลุ่มผู้สูงอายุที่อาจไม่คุ้นเคยกับการใช้งานระบบซับซ้อน

\textbf{ข้อแตกต่างจากระบบปัจจุบัน:}  
แพลตฟอร์มใหญ่ ๆ ใช้อัลกอริทึมซับซ้อนที่อาจไม่เหมาะกับผู้ใช้สูงวัย 
โครงงานนี้เลือกใช้วิธีที่เข้าใจง่าย ผ่านแบบสอบถามสั้น ๆ เพื่อสร้างการแนะนำสินค้า

\section{การชำระเงินดิจิทัลและระบบความปลอดภัย (Digital Payment and Security)}
เนื่องจากโครงการเกี่ยวข้องกับการซื้อขายออนไลน์ 
จึงต้องอาศัยความรู้ด้านระบบการชำระเงินดิจิทัล เช่น การโอนเงินผ่าน PromptPay \cite{promptpay2021} 
ซึ่งเป็นช่องทางที่สะดวกและได้รับความนิยมในประเทศไทย  
นอกจากนี้ยังต้องพิจารณาเรื่องความปลอดภัย (Security) \cite{pressman2014} 
ทั้งในการยืนยันตัวตนของผู้ใช้ และการปกป้องข้อมูลส่วนบุคคลตามมาตรฐานที่เกี่ยวข้อง

\textbf{ข้อแตกต่างจากระบบปัจจุบัน:}  
ระบบใหญ่รองรับการจ่ายเงินหลายช่องทาง แต่ซับซ้อนสำหรับผู้สูงอายุ 
โครงงานนี้จะเน้น PromptPay ที่ใช้ง่ายและคุ้นเคยในสังคมไทย

\section{การใช้รูปภาพและมัลติมีเดีย (Figures and Multimedia)}
ในการนำเสนอข้อมูล ผลการออกแบบ และผลการทดลอง \cite{pressman2014} 
จำเป็นต้องใช้รูปภาพ แผนภาพ และตาราง เพื่อสื่อความหมายได้ชัดเจน 
โดยในรายงานจะมีการอ้างอิงรูป (Figure) และตาราง (Table) ตามมาตรฐาน {\LaTeX} 
เช่นการใช้ \verb.\label. และ \verb.\ref. เพื่อเชื่อมโยงการอ้างอิงโดยอัตโนมัติ

\section{\ifenglish%
\ifcpe CPE \else ISNE \fi knowledge used, applied, or integrated in this project
\else%
ความรู้ตามหลักสูตรซึ่งถูกนำมาใช้หรือบูรณาการในโครงงาน
\fi
}
ความรู้ตามหลักสูตรวิศวกรรมคอมพิวเตอร์ที่ถูกนำมาใช้ในโครงการนี้ ได้แก่
\begin{itemize}
    \item การพัฒนาเว็บไซต์และเว็บแอปพลิเคชัน (Web Programming, Software Engineering) \cite{pressman2014}
    \item การออกแบบฐานข้อมูล (Database Systems)
    \item การวิเคราะห์และออกแบบระบบ (System Analysis and Design)
    \item ความรู้ด้านเครือข่ายคอมพิวเตอร์และความปลอดภัย (Computer Networks, Security) \cite{pressman2014}
\end{itemize}

\section{\ifenglish%
Extracurricular knowledge used, applied, or integrated in this project
\else%
ความรู้นอกหลักสูตรซึ่งถูกนำมาใช้หรือบูรณาการในโครงงาน
\fi
}
นอกเหนือจากความรู้ในหลักสูตรแล้ว โครงการนี้ยังอาศัยความรู้ที่เรียนรู้เพิ่มเติมด้วยตนเอง เช่น
\begin{itemize}
    \item การออกแบบประสบการณ์ผู้ใช้ (User Experience Design - UX) \cite{pressman2014}
    \item การใช้เครื่องมือจัดการโค้ดและทีมงาน (Git, GitHub, Docker)
    \item การใช้ระบบการชำระเงินดิจิทัล (PromptPay Integration) \cite{promptpay2021}
    \item การศึกษาแนวโน้มตลาด E-Commerce และพฤติกรรมผู้บริโภค \cite{otop2020}
\end{itemize}
ความรู้เหล่านี้ช่วยเสริมให้โครงการสามารถตอบโจทย์การใช้งานจริง 
และสอดคล้องกับความต้องการของผู้ใช้เป้าหมาย
